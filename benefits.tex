The potential benefits of shared modeling and virtual verification fall in two categories:
testing efficiency and collaborative advantages.

\subsection{Testing Efficiency}
\subsubsection{Earlier Software Integration Testing}
The main goal of virtual verification is to provide a means for testing the integration of software before the specific ECU/MCU hardware exists or before it is made available to the OEM.
%"Much earlier with virtual HW [...] Benefit for the OEM as testing could be done 1 year ahead"
%— Business Manager (Tier-2)
Earlier testing could also prove very valuable by identifying issues with the microcontroller specification if it is still evolving.
\begin{quote}
"[...] the specification of the MCU might be changing and therefore the model is still useful to find issues in the target embedded SW as well as with the specification of the MCU. Economically, the initial investment should not be too large as the models are not required to be highly accurate because there still might be many bugs in the MCU specification. As long as SW development is progressing than there is significant benefit even if the model is not initially high fidelity."
-- CEO (Tool Vendor 1)
\end{quote}

\subsubsection{Easier Testing}
Sharing microcontroller models for virtual verification may potentially have very positive effects on testing. A virtual testing environment is generally easier to setup than a corresponding HW based testing environment (e.g., hardware in the loop (HIL) systems), potentially leading to significant cost savings. HW test benches are typically a heavily shared resource resulting in scheduling conflicts and other logistic difficulties. If models with sufficient fidelity were available, it would reduce the need to build and physically access hardware-based test benches, potentially reducing scheduling problems. Using ECU and MCU models also increases the visibility and controllability of the testing process.

%de-quoted
Testing in the virtual environment is easier than the hardware environment because of the increased visibility of the microprocessor’s resources (e.g., Timing Processing Unit)
%— Business Manager (Tier-2)
Virtual boards also support parallelization, allowing for many tests to be executed in parallel assuming the host computers have enough processing power. This could provide additional cost and time savings when compared to running tests on a physical board.
%"A simulator although slower than real time can actually be used in a much quicker fashion than a real vehicle bench to reproduce the problem."
%— CEO (Tool Vendor 1)

\subsubsection{Uncovering More Faults}
Virtual testing will allow engineers to monitor, control, and observe the testing environment in a more comprehensive manner than currently possible on physical hardware. This includes easier ways to inject faults so that more thorough testing can be accomplished.
%These technologies could enable fault injection because they provide more visibility and controllability in the ECU HW resources. This could be done using the models while it can’t be done in general using the real HW.
%— Embedded Software/Tool Expert (Tier-1)
Thus, a virtual testing environment could be a valuable asset that allows for the defining and maintaining of tools for fault injection and other advanced testing procedures.

\subsubsection{HW Dependent Regression Testing}
Virtual testing would simplify SW regression tests on multiple (virtual) hardware platforms that do not need to be maintained across the entire product lifecycle (assuming the models used for virtual testing have sufficient fidelity and reliability). This could provide a cost-efficient way to ensure backward-compatibility for software patches, or evolve software components in a product family.
%"Virtual benches could be useful except for target independent functional verification."
%— Software Development Leader 2 (OEM)
%Note: Software components must also be tested independently from the hardware platform, which is sufficiently addressed by existing software in the loop (SIL) and model in the loop (MIL) verification mechanisms.

\subsubsection{Repeatable Testing}
Relying on virtual regression testing would significantly improve repeatability.
%"[Virtual] Regression testing is much more powerful and repeatable using virtual testing platforms"
%— Business Manager (Tier- 2)
Having easily repeatable test procedures would allow for new and more efficient ways of developing (potentially target dependent) software. Running regression tests more often in a defined environment could make it easier to find bugs earlier and reduce expensive recalls.
\begin{quote}
"In the traditional auto process typically it may take 6 months between when the SW bug is created and when the SW bug is detected! If you can switch from 6 months to 1 week cycle! This could be a breakthrough for the OEM."
— Business Manager (Tier-2)
\end{quote}

\subsection{Collaborative Advantages}
%In addition to increasing testing efficiency, shared modelling and virtual verification promises to support collaborations throughout the automotive ecosystem.

\subsubsection{Within the OEM}
The ability to do component level simulation and virtual verification across hundreds of developers promises to enable new ways of working and accelerate (continuous) software integration.
%"Success stories include [...] component level simulations across 100+ users in [OEM]"
%— Software Development Leader (OEM)
Increasing design complexity makes it necessary for large numbers of engineers to have access to shared systems and engineering resources. A shared modeling and virtual verification capability will help address this necessity beyond what is typically in place today.

\subsubsection{Within the Automotive Software Value Chain}
Shared modeling and virtual verification promises to improve and accelerate how suppliers develop software. For example, if suppliers are encouraged to provide models and virtual assets early in the process to their customers (other suppliers and OEMs), then they would likely embrace a model-first approach, where models of their product are delivered first, integrated and verified upstream, before the development of hardware and low level software starts. This would also benefit suppliers because it would encourage them to utilize model-driven development and virtual verification for their own internal processes.
%"One internal use case of shared modeling and virtual verification is HW verification at the RTL level in conjunction with emulation boxes."
%— Marketing Director (Tool Vendor 2)
By improving the awareness and the attractiveness of virtual verification technology across the complete ecosystem, the overall process and product quality can be improved, enabling Tier-1 and Tier-2 suppliers to fully benefit from model-driven engineering.

\subsubsection{Across the Automotive Ecosystem}
In addition to facilitating collaboration between ecosystem actors, a shared modeling and virtual verification capability may also help facilitate the exchange of conceptual knowledge. This would provide feedback on system and SW design and integration before hardware is available, significantly shortening the feedback cycle and enabling effective collaborative engineering much earlier in the design process.

% de-quoted
Tier2 use cases are [either] internal or external and enable customers with three different use cases: (i) Internally, [the Tier2 enables their] own internal MCAL SW development (ii) Also internally, [the Tier2 enables] HW verification at the RTL level in conjunction with emulation boxes. Although these models are not meant to be used as SW development platforms, these models could be the starting point for it. (iii) Externally, [the Tier2 enables collaboration] with their customers (typically Tier1s).
%— Marketing Director (Tool Vendor 2)

From the perspective of an OEM, these three use cases imply certain benefits: Wide adoption of model-driven development practice among the Tier-2 suppliers could increase the modeling expertise throughout the automotive ecosystem. The promise of re-using internally used models for verification better integration and alignment of model-driven development flows between Tier-1 and Tier-2 suppliers could decrease lead-time because: (a) it mitigates the risk of potential delays in the delivery of actual ECU hardware and (b) it mitigates delays in developing the low-level software (which could progress even in the absence of the actual hardware).
