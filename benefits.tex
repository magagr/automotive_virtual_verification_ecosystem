% !TEX root = main.tex
\section{Benefits} \label{sec:benefits}
Our respondents indicate that %the 
benefits of shared modeling and virtual verification fall in two categories:
testing efficiency and collaborative advantages \emph{throughout the ecosystem}.

Shared modeling and virtual verification increases testing efficiency by allowing earlier testing activities, increasing repeatability of tests, and uncovering more faults. 

%\begin{itemize}
   % \item Testing earlier.
    %\item Increasing repeatability of tests.
    %\item Uncovering more faults.
%\end{itemize}

%\subsection{Collaborative Advantages}
% \magnus{Ecosystem emphasis needed below}
In addition, % to increasing testing efficiency, 
our respondents indicate that shared modeling and virtual verification promises advantages related to the co-opetitive nature of software ecosystems (similar to co-opetition in open source software ecosystems \cite{Agerfalk2008}).
%
% The ability to do component level simulation and virtual verification across hundreds of developers promises to enable new ways of working and accelerate (continuous) software integration.
% %"Success stories include [...] component level simulations across 100+ users in [OEM]"
% %— Software Development Leader (OEM)
% Increasing design complexity makes it necessary for large numbers of engineers to have access to shared systems and engineering resources. A shared modeling and virtual verification capability will help address this necessity beyond what is typically in place today.
%
%Shared modeling and virtual verification promises to improve and accelerate how suppliers develop software.
For example, if suppliers are encouraged to provide models and virtual assets to their customers (other suppliers, OEMs, etc.) early in the design process, then they would likely embrace a model-first approach where product models are delivered and integrated/verified upstream before the development of hardware and low-level software starts.
This would also benefit suppliers because it would encourage them to utilize model-driven development and virtual verification for their own internal processes.
By improving awareness and attractiveness of virtual verification technology across the complete ecosystem, the overall process and product quality can be improved, enabling Tier-1 and Tier-2 suppliers to fully benefit from model-driven engineering.

%\subsubsection{Across the Automotive Ecosystem}
Beyond strengthening collaboration between ecosystem actors, a shared modeling and virtual verification capability helps 
facilitate the exchange of conceptual knowledge. 
%This provides feedback on system and software design and integration before hardware is available, significantly shortening the feedback cycle and enabling effective collaborative engineering much earlier in the design process.
This allows feedback on design and integration, even before hardware is available, and enables effective collaborative engineering much earlier in the design process.


% \begin{quote}
% "Tier-2 use cases are [either] internal or external and enable customers with three different use cases: (i) Internally, [the Tier-2 enables their] own internal MCAL SW development (ii) Also internally, [the Tier-2 enables] HW verification at the RTL level in conjunction with emulation boxes. Although these models are not meant to be used as SW development platforms, these models could be the starting point for it. (iii) Externally, [the Tier-2 enables collaboration] with their customers (typically Tier-1s)."
% -- Marketing Director (Tool Vendor 2)
% \end{quote}

%From the perspective of an OEM, these three use cases imply certain benefits:
% Wide adoption of model-driven development practice among the Tier-2 suppliers could increase the modeling expertise throughout the automotive ecosystem.
% The promise of re-using internally used models for verification, better integration, and alignment of model-driven development flows between
% %Tier-1 and Tier-2
% suppliers could decrease lead-time because:
% (a) it mitigates the risk of potential delays in the delivery of actual ECU hardware and 
% (b) it mitigates delays in developing the low-level software (which could progress even in the absence of the actual hardware).
