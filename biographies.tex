% biography section
% 
% If you have an EPS/PDF photo (graphicx package needed) extra braces are
% needed around the contents of the optional argument to biography to prevent
% the LaTeX parser from getting confused when it sees the complicated
% \includegraphics command within an optional argument. (You could create
% your own custom macro containing the \includegraphics command to make things
% simpler here.)
%\begin{IEEEbiography}[{\includegraphics[width=1in,height=1.25in,clip,keepaspectratio]{mshell}}]{Michael Shell}
% or if you just want to reserve a space for a photo:

% Articles should be accompanied by a short biographical sketch.
% It should contain, in the following order,
% your current position and technical interests,
% prior applicable professional experience,
% education, professional affiliations, and address.
\begin{IEEEbiographynophoto}{S. Magnus Ågren}
is a PhD student at the Department of Computer Science and Engineering, Chalmers $\mid$ University of Gothenburg.
His main research interests are in software engineering: the intersection between software integration practices and software architecture, and systems of systems.
Prior to taking up his PhD position, he worked as a software engineer in the telecom industry, at Ascom Wireless Solutions, with embedded systems, IP telephony, and systems interoperability.
He received his M.Sc. in Computer Science and Engineering from Chalmers University of Technology in 2010.
\end{IEEEbiographynophoto}

\begin{IEEEbiography}{Eric Knauss}
is Associate Professor at the Department of Computer Science and Engineering, Chalmers $\mid$ University of Gothenburg.
His research focuses on managing requirements and related knowledge in large-scale, distributed software and systems development: Requirements Engineering, Software Ecosystems, Global Software Development, and Agile Methods.
He has co-authored more than 50 publications in journals and international conferences, has been on organization and program committees of leading conferences in the field, and is a reviewer for top ranked journals.
In national and international research projects, he collaborates with companies such as Bosch, Ericsson, Grundfos, GM, IBM, Siemens, TetraPak, and Volvo.
More information at \url{https://oerich.wordpress.com}
\end{IEEEbiography}

\vfill

\begin{IEEEbiography}{Paolo Giusto}
is a Model-Based Systems Technical Leader at GM R\&D in Sunnyvale, California.
In his current function, he focuses on processes, methods, and tools for the accelerated development of E/E Control and SW Architectures.
Prior to joining GM, Paolo has worked at Cadence Design Systems in San Jose, California, from 1996 to  2004, where he has held several positions including product manager, and technical marketing director for the automotive tool suite.
Prior to joining Cadence, Paolo has worked at Magneti Marelli (FIAT group) in Turin, Italy, from 1991 until 1996, in the area of methodologies and tools for HW/SW co-design of embedded systems.
While at Magneti Marelli, Paolo was a visiting industrial fellow at UC Berkeley-EECS Department in California, from 1992 to 1994.
Prior to joining Magneti Marelli, Paolo served in the Italian Army from 1989 until 1990 where he was responsible for the data base management of the administrative branch.
He has published 50+ technical papers, co-authored three books, has two patents, and one patent pending.
Paolo holds a Master Degree in Computer Science and a MBA in Finance.
\end{IEEEbiography}

\begin{IEEEbiography}{Grant Soremekun}
is currently a Staff Researcher at General Motors R\&D.
In his current role, Grant works on identifying and evaluating model-based engineering (MBE) processes and tools for improved automotive system design, with a primary focus on electrical and control systems.
He has over 20 years of experience developing and applying MBE software for complex system design.
He has worked with some of the world’s most successful companies and government and academic institutions to implement simulation workflow automation and optimization strategies for designing aircraft, spacecraft, automobile, composite structure, power electronic, commercial building, and agricultural systems.
He also has several years of experience as a technical business development manager and sales representative for commercial MBE software solutions.
Grant holds a B.Sc. in aerospace engineering and an M.Sc. in engineering mechanics from Virginia Tech.
\end{IEEEbiography}

\begin{IEEEbiography}{Rogardt Heldal}
is Professor at the Bergen University College, Norway, Faculty of Engineering and Business Administration, and he is also Associate Professor (on leave) at Chalmers $\mid$ University of Gothenburg, Sweden, Department of Computer Science and Engineering.
His research topics are mainly in software engineering, software architecture modeling, software processes and empirical studies.
He has co-authored more than 70 publications in journals and international conferences and workshops on these themes.
He has been on the program committees for several conferences, and is a reviewer for top journals in the software engineering domain.
He is active in International and National projects.
He is an ISERN member.
In his research activity he has collaborated with several industries such as Volvo Cars, Volvo AB, Ericsson, Grundfos, Axis Communications and Saab.
He got his PhD in 2000 at the Chalmers University of Technology.
\end{IEEEbiography}

\vfill

\begin{IEEEbiography}{Daniela Damian}
is a Professor of Software Engineering in University of Victoria’s Department of Computer Science, where she leads research in the Software Engineering Global interAction Laboratory (SEGAL, \url{thesegalgroup.org}).
Her research interests include Software Engineering, Requirements Engineering, Computer-Supported Cooperative Work and Empirical Software Engineering. Her recent work has studied the developers’ socio-technical coordination in geographically distributed software projects, as well as stakeholder management in large software ecosystems.
Daniela’s research methodologies involve extensive field work and in-situ studies of software teams through collaborations with industrial partners such as IBM, General Motors, Siemens and Dell.
Daniela has served on the program committee boards of several software engineering conferences, as well as on the editorial boards of Transactions on Software Engineering, the Journal of Requirements Engineering, the Journal of Empirical Software Engineering, and the Journal of Software and Systems.
\end{IEEEbiography}

% insert where needed to balance the two columns on the last page with
% biographies
%\newpage

% You can push biographies down or up by placing
% a \vfill before or after them. The appropriate
% use of \vfill depends on what kind of text is
% on the last page and whether or not the columns
% are being equalized.

% Can be used to pull up biographies so that the bottom of the last one
% is flush with the other column.
%\enlargethispage{-5in}