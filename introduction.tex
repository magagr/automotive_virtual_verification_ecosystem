% !TEX root = main.tex

% The very first letter is a 2 line initial drop letter followed
% by the rest of the first word in caps (small caps for compsoc).
% 
% form to use if the first word consists of a single letter:
% \IEEEPARstart{A}{demo} file is ....
% 
% form to use if you need the single drop letter followed by
% normal text (unknown if ever used by the IEEE):
% \IEEEPARstart{A}{}demo file is ....
% 
% Some journals put the first two words in caps:
% \IEEEPARstart{T}{his demo} file is ....
% 
% Here we have the typical use of a "T" for an initial drop letter
% and "HIS" in caps to complete the first word.
% You must have at least 2 lines in the paragraph with the drop letter
% (should never be an issue)

\IEEEPARstart{T}{he} development of automotive control architectures is increasingly dependent on the use of modeling and simulation technologies. Historically, these techniques are often applied reactively and only later in the development process. Automotive OEMs (original equipment manufacturers) typically act mainly as
 systems  integrators of pre-allocated
 functionality to  ECU (electronic control unit) hardware. In today's today's economic climate this practice is however becoming less sustainable. 
 
 Virtual verification based on shared models (see Sidebar \ref{bar:tech}) is a strategy providing the ability to obtain feedback about design decisions much earlier in the development process. Although more desirable, virtual verification of models represents a major shift in the automotive domain as it demands flexible and collaborative approaches within the complex model-driven ecosystem including actors such as model providers, tool providers, model qualifiers and OEMs (see Sidebar 1). 
 In this paper we analyze the impediments and enablers \ugh{to the adoption}\eric{would prefer to give it less of a technology diffusion/adoption/transfer flavour and more an ecosystem flavour. E.g. by considering the surrounding ecosystem we obtain a more holistic view}\magnus{I concur} of virtual verification by the relevant stakeholders in the automotive ecosystem. 
% The primary consequence of these problems is the OEM's struggle to design systems of ever-increasing complexity due to the growing number of errors,
% that often originate early in the conceptual and preliminary design phases, but aren't discovered until later during the hardware/software integration phase.
% Furthermore, the development speed of hardware and software components
% %(e.g., microcontrollers and software components)
% are limited by the supplier's own hardware and software development cycles.
% Thus, the integration of software and hardware can only be tested very late in the development cycle,
% further contributing to the possibility of costly software bugs and design errors.
% This leads to increased production delays and costly vehicle recalls, that significantly reduce product quality and profit margins.
% Finally, using component models and corresponding analysis and simulation tools to support the early verification and integration of control systems has not yet been widely accepted by OEMs
% -- or the automotive design community in general --
% as part of a standard development process.
% Currently, standard development processes are driven primarily by the exchange of tangible assets (ECUs, microcontrollers) and software IPs
% %(e.g., low lever device drivers),
% and not models of it, therefore preventing any opportunity of early integration.
%Understanding the underlying ecosystem of cross-organizational collaborations will allow us to properly articulate the benefits, impediments, and enablers of designing electrical control architectures using shared modeling and virtual verification methodologies.

%\subsection{Availability of models vs. availability of hardware}
%Virtual verification based on shared models is a strategy to manage the cone of uncertainty \cite{Boehm1981}, 
%which states that uncertainty is gradually reduced as development progresses,
%more information about the system becomes available,
%and system integration aspects can be tested.
%Many decisions need to be made at the beginning of the development process, though, when uncertainty is the greatest.
%Thus, the ability to give early feedback about design decisions is an important asset.
%Consequently, any model characterizing hardware performance before the actual silicon hardware exists can be valuable in helping to develop the embedded software earlier.
%Once the hardware exists, models need to depict the hardware as closely as possible.
%A lack of accuracy is then hard to compensate by additional value of models over the real hardware, such as easy, location independent availability and scalability.

% Several R\&D and engineering teams,
% at a large North American OEM,
% have been addressing the organizational and technical challenges related to the mainstream adoption of modeling and simulation technologies, for the virtual development and integration of electrical control architectures, to reduce the risk of late design changes.
% The main area of progress has been the adoption of hardware-in-the-loop (HIL) benches,
% coupling real time models of plants and of the rest-of-the-vehicle serial data traffic with real loads, sensors, and controllers.
% Other areas of progress include software-in-the-loop (SIL) technologies, used to simulate the control behavior coupled with simpler real time models of plants.
% Finally, high-fidelity models of microprocessors running the real target controller code have been investigated.
% While HIL benches are a mature area at the OEM, other areas exhibit technical and organizational challenges of various nature.
%In this article, we discuss some of the benefits, challenges, and enablers of modeling and simulation technologies for electrical control architecture,
%by analyzing the high-fidelity modeling and simulation of processor technologies.
%We then generalize our findings to the more general area of modeling and simulation technologies for virtual development and integration.
%We believe that the work described in this article is of significance to any one
%at the OEM
%who is facing challenges, trying to "cross the chasm" from early niche adoption to main stream.
%By analyzing benefits, enablers, and impediments to adoption by means of interviews and workshops with relevant stakeholders in the ecosystem, we are able to identify potential ways to overcome the resistance to change, or at least possible ways to sell the value proposition.

%Model-based systems engineering aims to capture this information within a model that is managed throughout the product lifecycle. It is promising to extend these system models with relevant models on lower abstraction levels such as Microprocessor models that describe how ECUs are implemented. If connected, they would provide a bottom-up model for systems engineering, covering the automotive value chain (i.e. the process and activities that add value to an automotive system across OEM and suppliers), and allowing to reason about system aspects in a holistic way.

%We argue that leveraging these potentially large benefits requires bootstrapping beneficial relationships in an automotive model-driven software ecosystem, based on a virtualization technology platform (see Sidebar \ref{bar:ecosys}). 
%This paper presents benefits, enablers, and impediments to adoption by relevant stakeholders in the ecosystem; information useful when articulating a value proposition for overcoming  resistance to change.
%, or at least possible ways to sell the value proposition.
%we provide an overview of the technology area of focus.
 Drawing on our experience in the North American and European auto industries over the last decade, we report here on a focused investigation of the virtual verification ecosystem 
with key stakeholders at five representative companies, including a large North American OEM, its suppliers, and tool vendors. Data from our initial semi-structured interviews with these stakeholders was analyzed for themes to identify benefits, impediments as well as enablers of adopting shared modeling and virtual verification in the automotive model-driven ecosystem. We validated and improved our understanding of our research results in a follow-up 2-day workshop with six OEM engineers and managers. 

%Tie into impediments and enablers