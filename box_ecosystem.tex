\small
The virtual platform technology ecosystem involves suppliers of models and tools, suppliers of hardware and software IP (intellectual property), users, etc., and is described in \cite{Knauss2014d}.
\magnus{Should we elaborate here on the degree to which this ecosystem is currently established?}
A potential supply network of this ecosystem is depicted above (based on the Software Ecosystem Modeling notation, \cite{Boucharas2009}).

The ecosystem is complex and may also involve the creation of new roles and responsibilities, e.g. the standardization of the microprocessor model certification process for accuracy and performance.
Ecosystem roles and responsibilities are similar in nature to those found in the ECU hardware ecosystem because there are mixed hardware and software IPs.
For hardware IP, the microprocessor core is often provided by a Tier-3 supplier who, in turn, provides this component to a Tier-2 microprocessor provider
The same scenario applies to software IPs, with platform software modules provided by one vendor, device drivers provided by another vendor, and the microcontroller abstraction layer provided by the chip provider.
On top of these software layers, an OEM performs the overall integration.
%As expected the ecosystem here is quite complex.

%A potential ecosystem for virtual platforms is \del{shown} \ins{}.\eric{Perhaps move these two sentences to the top or after the first sentence?}
The degree of granularity and the integration levels of the different IPs determines its complexity (e.g., integration of peripheral models into a microprocessor model, integration of microprocessor models and ASICs into an ECU model).
%As described earlier, other models/tools in addition to the virtual chipset simulator need to be integrated to accurately simulate the full functionality of the controller and configure the software image for testing via tools such as INCA.

\magnus{Describe technology adoption and add reference here?}