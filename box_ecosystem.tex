The virtual platform technology ecosystem involves the model/tool suppliers, software/hardware IP suppliers, users, etc., and is described in \cite{Knauss2014d}.
The ecosystem is complex and may also involve the creation of new roles and responsibilities,
e.g., the standardization of the microprocessor model certification process  for accuracy and performance.
Ecosystem roles and responsibilities are similar in nature to those found in the ECU hardware ecosystem, because there are mixed hardware and software IPs.
In the hardware IP case, often times the microprocessor core provider is a Tier-3 supplier providing this device to a microprocessor provider, a Tier-2.
The same scenario applies to software IPs, with platform software modules sold by one vendor, device drivers by another vendor, and the microcontroller abstraction layer provided by the chip provider.
On top of these software layers, an OEM performs the overall integration.
%As expected the ecosystem here is quite complex.

A potential ecosystem for virtual platforms is shown.
The degree of granularity and the integration levels of the different IPs,
e.g. integration of peripheral models into a microprocessor model, integration of microprocessor models and ASICs into an ECU model,
determines its complexity.
%As described earlier, other models/tools in addition to the virtual chipset simulator need to be integrated to accurately simulate the full functionality of the controller and configure the software image for testing via tools such as INCA.
