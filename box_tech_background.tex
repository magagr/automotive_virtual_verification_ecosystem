Hardware test benches, or HIL benches (hardware-in-the-loop), include one or more controller PCBs (printed circuit board) hosting the non-production ready ECU running the control software, a custom and fast hardware box running models of the controlled plant (e.g., the engine), the rest of the vehicle control architecture, the vehicle dynamics and the environment (e.g, road conditions), all synchronized in real time with the ECUs, plus a variety of other tools for calibrating the control parameters, software debuggers, etc., hosted by a dedicated computer.

Virtual platform technology is a key aspect of the ecosystem for model sharing and virtual verification,
as it extends concepts such as HIL and SIL (software-in-the-loop) benches to include microprocessor models.
This enables early software integration for OEMs,
as well as early software development of basic software by Tier-1 suppliers.
Enabling the parallel usage of these processor models among OEMs and their suppliers can achieve parallel rather than sequential software development,
whereby synchronization points are established for integration tasks and feedback loops between OEMs and suppliers.

The Virtual platform technology consists of two parts, a virtual chipset and a virtual platform for software development, integration, and testing.
The virtual chipset consists of a high-fidelity microprocessor model,
including peripherals and other resources that execute the target embedded software.
Extending the virtual chipset with additional models of off-chip devices essentially %\chg{realizes}
{completes} the concept of a virtual controller.
External models, similar to those required for HIL systems, are also required for the virtual controller to emulate the behavior of the real controller's functionality and performance.
These include models representing incoming serial data traffic,
plant models (e.g., an engine), simulating ASICs running microcode, interfacing high power interfaces such as inductors to the controlled loads (e.g., electric motors, fuel injectors),
vehicle dynamics, environment, etc.
The degree of modeling around the virtual chipset depends on the use case at hand.
In order to perform a full functional and performance verification of the controller, all the aforementioned models are needed.
For use cases involving only low-level software development, an open loop model providing hardware IOs or serial data traffic may be sufficient.
Finally, tools for configuring software via calibrations and for software variable/signal recording may also be needed.
The integration of several virtual chipsets may be necessary for sub-system and complex ECU software verification, and may require high-performance computing platforms to achieve acceptable simulation runtimes.