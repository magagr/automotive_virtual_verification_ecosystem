The focus of this article is the study of ecosystems for control architecture models and simulation technologies.
%We aim at understanding benefits, impediments, and enablers of modeling and simulation technologies based upon a scenario we analyze in detail and then generalize the findings of. Throughout the report, and based upon the interviews and the workshop we have hosted, we conclude that:
Such ecosystems
%for models and simulation technologies
may play a major role both in impeding and enabling their large scale adoption in development processes.

Modeling and Simulation technical advantages and business benefits may not be considered sufficient to justify their deployment as OEM organizational challenges may prevent it (e.g., lack of HRs).

Technical impediments and enablers are somewhat disconnected from other impediments and enablers. If there is an engineering need, based on business related enablers, a way to address the technical challenges will be found. In particular, we did not identify any insurmountable technical impediments for the deployment. However, we do recognize that complexity, variance, interoperability, scale, and similar impediments will be encountered and need to be addressed.

With respect to the impediments, Figure 1 shows an undesirable circular dependency: Since there is little or no demand for high fidelity modeling technology, there are limited HRs available, which leads to no (widespread) adoption. Consequently, there is no clear value proposition for widespread adoption of virtual platform technology, therefore making it difficult to generate enough demand.

The Lack of HRs is further increased by a hardware-centric, conservative culture that is accustomed to use Hardware in the loop benches with physical controllers (indeed mature technologies). In addition, gaps in the OEM virtual development tool global strategy makes it difficult to overcome the lack of widespread demand of modeling and simulation technology

With respect to the enablers, we see a clear value proposition as crucial to breaking the circular dependency, and creating more demand. This will be helped by clearly articulating customer value, and by properly scoping use cases based on a well understood stakeholder landscape. These enablers also need to be mapped to a company strategy in order to facilitate wider adoption. 

A suitable scope and increased transparency will in addition mitigate the lack of global strategy. The foundation of this is sharing knowledge about what works in this area and which concrete benefits have already been achieved as well as by combining related knowledge from different initiatives. This will encourage grass-root adoption (e.g., from visionary engineers) and may in turn trigger wider adoption and then resources toward mainstream adoption. Transparence about benefits will also help to mitigate the cultural impediments.

% Finally, we believe that overcoming these impediments will play a key role in increasing the competitive advantage of the OEM in the market place, as it will allow the OEM to reduce control system development lead time and costs (pre- and post-sales).
