The focus of this paper is 
the ecosystem for model sharing and virtual verification.
As the emerging ecosystem is still in its early stage,
we share a map of impediments and enablers that we found most useful to plan the activities needed to bring about the technical and collaborative benefits.
Specifically, we recommend to:
%Actions:
\begin{itemize}
    \item Choose starting use cases, to clarify the fidelity needs.
    \item Define the value resulting from solving the chosen use case.
    \item Bootstrap the ecosystem by sharing knowledge and clarifying the actor -- role mapping \cite{kilamo2012proprietary} .
\end{itemize}

Our findings indicate that virtual verification is a promising approach,
that could improve automotive software development across organizational boundaries.



%We aim at understanding benefits, impediments, and enablers of modeling and simulation technologies based upon a scenario we analyze in detail and then generalize the findings of. Throughout the report, and based upon the interviews and the workshop we have hosted, we conclude that:
% Such ecosystems
% for models and simulation technologies
% may play a major role both in impeding and enabling their large scale adoption in development processes.

%Modeling and Simulation technical advantages and business benefits may not be considered sufficient to justify their deployment as OEM organizational challenges may prevent it (e.g., lack of HRs).

% Finally, we believe that overcoming these impediments will play a key role in increasing the competitive advantage of the OEM in the market place, as it will allow the OEM to reduce control system development lead time and costs (pre- and post-sales).

%We find that a systematic, model-based approach could enforce good system engineering practices throughout the ecosystem, and offer a holistic view on value creating engineering activities. If further confirmed, these results could support establishing an integrated model-based engineering approach at the whole vehicle level. While our research focusses on accelerating software development - we see this as one important sub-use cases and a step towards supporting a complete model-based approach that would allow to capture all relevant information in a (federated) system engineering model (e.g. SysML) and link it to engineering analysis and implementation downstream tools.