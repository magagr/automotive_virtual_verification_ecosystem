\subsection{Technical}
Technical impediments and enablers are of secondary importance to business and organizational aspects.
If the organization is aligned on adopting virtual verification of ECUs,
any technical challenge should be surmountable, as long as resources are made available.
In particular, we did not identify any insurmountable technical impediments for deployment of virtual verification platforms.
Awareness of these impediments is still needed, since they will become more prevalent when deploying at scale.

%\subsubsection{Technical Impediments}
%we recognize that complexity, variance, interoperability, scale, and similar impediments will be encountered and need to be addressed.
Potential technical impediments include
lack of model interoperability,
models not kept up-to-date with evolving hardware,
lack of trust in model fidelity,
and lack of protection of intellectual property.

%\subsubsection{Technical Enablers}
Conversely, foreseeable technical enablers include:
A shared and standardized \emph{modeling platform},
\emph{reusable model templates} for handling product line variation points,
and, standardization of for example computational ECUs, at the \emph{system architecture} level.



\subsection{Business and Organization}
With respect to the impediments, figure \ref{fig:map} shows an undesirable circular dependency:
Since there is lack of demand for high fidelity modeling technology,
there's also a lack of global strategy for the use of virtual platform technology.
Without such a strategy, there's limited personnel available,
which leads to lack of adoption.
Consequently, there is no clear value proposition for widespread adoption of virtual platform technology, therefore making it difficult to generate enough demand.

% full circle before the below
The lack of adoption is further increased by a hardware-centric, conservative culture that is accustomed to use HIL benches with physical controllers -- indeed mature technologies.

% enablers
With respect to the enablers, we see a clear value proposition as crucial to breaking the circular dependency, and creating more demand.
This will be helped by clearly articulating customer value, and by properly scoping use cases based on a well understood ecosystem landscape.
These enablers also need to be mapped to a company strategy in order to facilitate wider adoption. 

A clear ecosystem landscape and knowledge sharing will in addition mitigate the lack of global strategy.
The foundation of this is sharing knowledge about what works in this area and which concrete benefits have already been achieved as well as by combining related knowledge from different initiatives.
This will encourage grass-root adoption (e.g., from visionary engineers) and may in turn trigger wider adoption and then resources toward mainstream adoption.
Transparency about benefits will also help to mitigate the cultural impediments.

