% !TEX root = main.tex
\section{Impediments and Enablers}\label{sec:impediments_and_enablers}

Despite this great promise, bootstrapping a technological platform and critical relationships in an ecosystem for shared modeling and virtual verification has proven difficult thus far.
Our data shows, in line with more general theories on technology diffusion \cite{rogers2010diffusion} %\magnus{Covering diffusion/adoption in the sidebar mights save us a few words}
and technology transfer \cite{gorschek2006model}, that any effort to introduce shared modeling and virtual verification needs to take into account the larger context of the automotive domain and its value-chains. 
Our analysis of the data collected from the interviews sought to identify aspects of business, social (organizational), and technical nature, as suggested by Christensen et al. \cite{christensen2014analysis} and Tamburri et al. \cite{tamburri2013uncovering} for reasoning about ecosystems. 
%\eric{Here, I believe we need to make the ecosystem aspect visible. We are not looking at this problem as a technology diffusion problem. We are looking at it as a software ecosystem topic. Thus, we find a different way to approach the known difficulties.}. 
Figure \ref{fig:map} describes the impediments and enablers that we found in the business and organizational areas. 
We omitted technical impediments from the figure, since our data suggests them to be of secondary importance:
They will only become tangible when deploying at scale.
%by our data (and thus omitted from the figure), they do become %\chg{prominent}
%{tangible} when deploying at scale.
%and \ugh{thus we describe them below}\eric{But they are now gone. I think we need them very shortly?}. \magnus{I suggest keeping them out, and dropping the forward reference.}

%\subsubsection{Technical Impediments}
%we recognize that complexity, variance, interoperability, scale, and similar impediments will be encountered and need to be addressed.
%\textbf{Potential technical impediments} include
%lack of \emph{model interoperability},
%challenges to \emph{keep models up-to-date} with evolving hardware,
%lack of \emph{trust in model fidelity},
%and lack of protection of \emph{intellectual property}.

%\subsubsection{Technical Enablers}
%Conversely, \textbf{foreseeable technical enablers} include a shared and standardized \emph{modeling platform},
%\emph{reusable model templates} for handling product line variation points,
%and increased standardization at the \emph{system architecture} level, for example of computational ECUs.



%\subsection{Business and Organization}
With respect to \textbf{business- and organization-related impediments}, Figure \ref{fig:map} shows an undesirable circular dependency:
since there is a \emph{lack of demand} for high-fidelity modeling technology,
there is also a \emph{lack of global strategy} for the use of virtual platform technology.
Without such a strategy, there is limited personnel available (\emph{lack of resources}),
which leads to \emph{lack of adoption}.
Consequently, there is no clear value proposition for widespread adoption of virtual platform technology, therefore making it difficult to generate enough demand.

% full circle before the below
The lack of adoption is further increased by a hardware-centric, conservative \emph{culture} that is accustomed to using HIL benches with physical controllers.

% enablers
As \textbf{business- and organization-related enablers}, %With respect to the enablers, 
we see a \emph{clear value proposition} as crucial to breaking the circular dependency and creating more demand.
This will be helped by clearly articulating customer value and by \emph{properly scoping use cases} based on a \emph{clear ecosystem landscape}.
These enablers also need to be mapped to an overall company strategy in order to facilitate wider adoption. 

A \emph{clear ecosystem landscape} and \emph{knowledge sharing} are essential enablers to mitigate the lack of global strategy and facilitate adoption of virtual verification.
Sharing knowledge about the ecosystem actors, their current capabilities, what benefits have already been achieved, and combining knowledge from different initiatives will encourage grassroots adoption (e.g., from visionary engineers). 
This may in turn trigger wider adoption
%\chg{and then resources toward mainstream adoption. Transparency about benefits will also help to mitigate the cultural impediments.}
as well as increased availability of resources to support mainstream adoption. 
Encouraging transparency about the potential benefits may also mitigate cultural impediments.

\subsection{Impediments in Context}
We concretize the impediments 
\emph{lack of global strategy} and \emph{lack of resources}
by describing a specific scenario:
Modern automotive MCUs carry several core computing units, including specific time processing units.
While the first class of resources execute traditional embedded software, the latter class executes specific micro code, scheduled on the timer, at a very high rate.
OEMs, in their quest to enable earlier software integration and testing on the target ECU, have started to internalize the development of the micro code for the time unit.
This is directly possible if the time unit is available.
If not, a model for the time unit, enabling execution of micro code for the actual time unit, can be a very effective strategy.
It is, however, unclear which ecosystem actor should pay for the development of the time unit model, as one of the Tool Vendor Marketing Director mentioned: 

\begin{quote}
"No one will ever have all the IPs\footnote{Intellectual property, i.e. creations of the mind, including industry designs, patents. Here: SW/HW components traded between ecosystem actors} at any given time. So new IPs will always have to be created/modeled, costing time and money. The question is: who will model it?"
%-- Tool Vendor Marketing Director
\end{quote}

Should the OEM pay, and then own and license the model, or should a tool supplier pay?
For a typical supplier, providing modeling support for users at the higher  abstraction level requires a major investment, and would depend on customers requesting -- and paying -- for it.
%Nor is it obvious, in terms of return on investment, why an OEM should bear the costs of deployment of the virtual environment for the development of the micro code, leading to a \emph{lack of resources}.
For an OEM, there is no obvious return of investment for bearing the costs of deployment of the virtual environment for the development of the micro code, leading to a \emph{lack of resources}.

% \subsubsection*{Lack of Resources \magnus{Example replaces}}
% While there may be some strong support within an OEM to pursue a global strategy, resources may be lacking especially if there is support from engineering without strong backing from leadership.

%\subsubsection*{Cultural Inertia}
Having mechanical roots, the automotive domain has traditionally had little 
%or no 
trust in virtual verification of software.
%while it has adopted CAE/CAD tools for plant virtual development, integration, and testing for over a decade.
Such \emph{cultural inertia} %Instead, there is 
shows as bias towards tangible ``real'' hardware assets.
Ideally, dedicated resources would support all modeling activities.
However, as an OEM Software Development Leader reported:

\begin{quote}
"Everybody is always focused on the 'current' development objectives.
Introducing new capabilities from a model-based engineering perspective and institutionalizing them is very time consuming, and people ‘have no time’." 
\end{quote}

% \subsubsection*{Lack of Adoption}
%In addition to a lack of a support organization, 
We found that \emph{lack of adoption} and \emph{demand} 
is also caused by lacking expertise and experience at all three ecosystem actors (OEM, Tier-1, Tier-2).
The use of virtual prototyping tools and models requires a high degree of understanding of the execution platform that needs to be modeled.
%
%In addition, engineers throughout the ecosystem are under tremendous pressure to complete work on time, leaving little or no opportunity to adopt new processes. Thus, it is difficult to overcome legacy processes throughout the automotive ecosystem, as well as to establish the ecosystem thinking that is required to effectively manage development throughout the value chain.
% Even within the OEM's engineering community, there is \emph{lack of demand} for such a virtual verification and model exchange platform.
% In a typical OEM, there is lack of agreement on whether such capabilities would  yield a positive ROI.
% For example, in-house component level software development aims at being hardware independent, thus software developers may see limited value in virtual verification.
%
It is also uncommon for an OEM and its suppliers to share models.
%(i.e. IP needs to be protected).
%While an OEM could ask for access to a high-fidelity model provided by a supplier, there is no clear understanding on why or whether this should be done.
%This lack of understanding might be because a significant amount of control SW development is performed in-house by the OEM.
As suppliers typically target multiple customers, their designs may have hidden functionality not in line with a particular customer’s data sheet,
adding difficulty to sharing models currently intended for internal use.

%A tendency to build instead of buy based on a protective mindset minimizes the need to use models from external partners.
%For similar reasons, there's reluctance to step away from known methodologies even though they are often antiquated.


\subsection{Enablers in Context}

%\ins{The biggest enabler mentioned in our interviews is the \emph{clear ecosystem landscape}, i.e. the knowledge baout the relationships...}

%\subsubsection*{Clearly Value Proposition}
As the primary users of a virtual environment for micro code development, OEMs should articulate a \emph{clear value proposition} (e.g. early software integration and testing before the time processing unit is available in hardware).

\begin{quote}
"Virtual prototyping means virtualizing everything in the development process, not only the MCU digital hardware.
The definition of success should not be how accurate the model is, it should be whether the project objectives have been achieved, e.g., software developed and tested early in the design process."
-- Tool Vendor CEO
\end{quote}

OEMs should also quantify the return on investment, both of bearing some costs of the model development, and of the deployment effort within existing and mature software development processes.
Only when the business aspects have been clearly stated and evaluated, and accepted by management, should deployment of the time unit model and related tool chains for micro code development commence.


% \subsubsection*{Clear Value Proposition}
% It is critical to define a compelling value proposition for investing in virtual prototyping models and tools.
% The value proposition changes depending upon the problem to be solved, however.
% For example, during the initial phases of software development,
% the value proposition is that
% integration and testing activities can start earlier in the design process.
% %providing an opportunity to optimize the software design, discover errors, etc.

% In order to make a sizable initial investment in virtual software development palatable to ecosystem actors,
% incentives would need to be provided by the OEM,
% both internally to foster good practice, and towards suppliers that could support virtual software development.
% These incentives could then spread throughout the value chain.

%\subsubsection*{Properly Scoped Use Cases}
To analyze tradeoffs between fidelity and speed, or functional and timing accuracy, \emph{properly scoped use cases} need to be developed and communicated.
Over-expectation on the required level of model fidelity can lead to a belief that virtual verification is unfeasible.
%\ugh{We recommend}\eric{I agree with DD, recommendations should go into the summary and recommendations section. However, this paragraph is meant to describe the enabler (properly scoped use case). Thus, the text must remain, but not formulated as a recommendation.} 
Instead, a use case should provide a realistic assessment of the minimal required accuracy 
%allow to assess the level of accuracyIt is a crucial enabler to investigate whether models really need to be highly accurate for their intended use, 
and %to offer 
constructive guidance for a cost/benefit tradeoff.
To realistically implement virtual platform technology, the scope must be manageable while also large enough to engage various aspects of the ecosystem (e.g., cross-organizational collaboration and new business models of ecosystem actors).

% \subsubsection*{Clear Ecosystem Landscape}
% % A clear ecosystem landscape would support the definition of a global strategy.
% Many impediments relate to missing competencies and resources throughout the automotive ecosystem. 
% Detailed knowledge about potential business models for various ecosystem actors (e.g. hardware suppliers, tool providers, model providers, see sidebar \ref{bar:ecosys}), is needed to define a suitable global strategy.

%\subsubsection*{Knowledge Sharing}
Smaller %proof of concept 
pilot projects that aim at showing the value proposition of virtual verification of software, have been and are under way in the automotive industry.
Examples include early software integration and testing, third party software %intellectual property 
validation before software integration, and end of development controller behavioral and performance testing.
The results so far are promising, yet, the major enabler to the widespread adoption of virtual verification across the automotive ecosystem is awareness through \emph{knowledge sharing} and establishing a \emph{clear ecosystem landscape}.

This can be achieved through establishing transparency about the virtual verification ecosystem, its challenges and difficulties, but also its benefits and successes. 
Pilot studies and increasing adoption by OEMs should include documenting the experience with suppliers who are able to support virtual verification, facilitating knowledge sharing throughout the ecosystem.
A pilot demonstrating a strong value chain that can support key use cases would be an effective enabler for further adoption.
% Moved from impediments
Specifically, the following knowledge needs to be shared:
\begin{itemize}
  \item By end users: model requirements (e.g. speed and fidelity) that allow virtual verification of software.
  \item By tool vendors: knowledge of their simulation technology.
  \item By OEMs: knowledge about their specific software customizations.
\end{itemize}